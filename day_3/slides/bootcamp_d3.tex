
\documentclass{beamer}

% There are many different themes available for Beamer. A comprehensive
% list with examples is given here:
% http://deic.uab.es/~iblanes/beamer_gallery/index_by_theme.html
% You can uncomment the themes below if you would like to use a different
% one:
%\usetheme{AnnArbor}
%\usetheme{Antibes}
%\usetheme{Bergen}
%\usetheme{Berkeley}
%\usetheme{Berlin}
%\usetheme{Boadilla}
%\usetheme{Frankfurt}
%\usetheme{boxes}
%\usetheme{CambridgeUS}
%\usetheme{Copenhagen}
%\usetheme{Darmstadt}
\usetheme{focus}
\usepackage{hologo}
%CREATES THE TABLE OF CONTENTS AT THE TOP
%\useoutertheme[subsection=false,shadow]{miniframes}
%\useinnertheme{default}
\usepackage{tikz}
\usepackage{url}
\usepackage{color}
\usepackage{longtable}
\usepackage{booktabs}
\usetikzlibrary{shapes.geometric, arrows, positioning, calc}
\tikzstyle{startStop} = [rectangle, rounded corners, minimum width=7cm, text width=11cm, text centered, minimum height=.5cm, draw=black]
\tikzstyle{io} = [circle, rounded corners, minimum width=1cm, text width=1.5cm, minimum height=.1, text centered, draw=black]
\tikzstyle{arrow} = [thick,->,>=stealth]
\usepackage{pdfpages}
\usepackage{graphicx}   % need for figures
\usepackage{adjustbox}
\usepackage{fontawesome}
\usepackage[absolute,overlay]{textpos}
%CHANGES COLOR TO GREEN
\usepackage{listings} 
\usepackage{color}

\definecolor{codegreen}{rgb}{0,0.6,0}
\definecolor{codegray}{rgb}{0.5,0.5,0.5}
\definecolor{codepurple}{rgb}{0.58,0,0.82}
\definecolor{backcolour}{rgb}{0.95,0.95,0.92}

\lstdefinestyle{mystyle}{
	backgroundcolor=\color{backcolour},   
	commentstyle=\color{codegreen},
	keywordstyle=\color{magenta},
	numberstyle=\tiny\color{codegray},
	stringstyle=\color{codepurple},
	basicstyle=\scriptsize,
	breakatwhitespace=true,         
	breaklines=true,                 
	captionpos=b,                    
	keepspaces=false,                 
	numbers=left,                    
	numbersep=5pt,                  
	showspaces=false,                
	showstringspaces=false,
	showtabs=false,                  
	tabsize=2
}
\hypersetup{
	colorlinks = true,
	linkcolor=brickred,   % color of internal links
	citecolor=brickred,   % color of links to bibliography
	urlcolor=brickred,    % color of external links
%	pagebackref=true,
%	implicit=false,
%	bookmarks=true,
	bookmarksopen=true,
	pdfdisplaydoctitle=true
}
\lstset{style=mystyle}
\definecolor{brickred}{rgb}{0.8, 0.25, 0.33}
\definecolor{mygreen}{cmyk}{0.82,0.11,1,0.25}
\setbeamertemplate{blocks}[rounded][shadow=true]
\addtobeamertemplate{block begin}{\pgfsetfillopacity{0.8}}{\pgfsetfillopacity{1}}
\setbeamercolor{structure}{fg=mygreen}
\setbeamercolor*{block title example}{fg=white,
	bg= white}
\setbeamercolor*{block body example}{fg= white,
	bg= white}
\usepackage[english]{babel}
\usepackage{hyperref}
\usepackage{dcolumn}
\usepackage{adjustbox}
\usepackage{multicol}
\usepackage{adjustbox}
\usepackage{amsmath}
\usepackage{tikz}
\usepackage[all,cmtip]{xy}
\tikzstyle{largeSquare} = [rectangle, rounded corners, minimum width=7cm, text width=9cm, minimum height=.5cm, draw=black]

\usetikzlibrary{shapes.geometric, arrows}
\tikzstyle{arrow}=[thick,->,>=stealth]
\beamertemplatenavigationsymbolsempty
\usepackage{subfloat}
\setbeamertemplate{headline}{}
\newcommand{\speechthis}[2]{
	\tikz[remember picture,baseline]{\node[anchor=base,inner sep=0,outer sep=0]%
		(#1) {\underline{#1}};\node[overlay,ellipse callout,fill=blue!50] 
		at ($(#1.north)+(-.5cm,0.8cm)$) {#2};}%
}%
%\usetheme{Goettingen}
%\usetheme{Hannover}
%\usetheme{Ilmenau}
%\usetheme{JuanLesPins}
%\usetheme{Luebeck}
%\usetheme{Madrid}
%\usetheme{Malmoe}
%\usetheme{Marburg}
%\usetheme{Montpellier}
%\usetheme{PaloAlto}
%\usetheme{Pittsburgh}
%\usetheme{Rochester}
%\usetheme{Singapore}
%\usetheme{Szeged}
%\usetheme{Warsaw}
\setbeamercolor{button}{bg=mygreen,fg=white}
%\setbeamercovered{transparent=25}
\setbeamercovered{invisible}

\usecolortheme{beaver}

\definecolor{codegreen}{rgb}{0,0.6,0}
\definecolor{codegray}{rgb}{0.5,0.5,0.5}
\definecolor{codepurple}{rgb}{0.58,0,0.82}
\definecolor{backcolour}{rgb}{0.95,0.95,0.92}

\lstdefinestyle{mystyle}{
	backgroundcolor=\color{backcolour},   
	commentstyle=\color{codegreen},
	keywordstyle=\color{magenta},
	numberstyle=\tiny\color{codegray},
	stringstyle=\color{codepurple},
	basicstyle=\footnotesize,
	breakatwhitespace=false,         
	breaklines=true,                 
	captionpos=b,                    
	keepspaces=true,                 
	numbers=left,                    
	numbersep=5pt,                  
	showspaces=false,                
	showstringspaces=false,
	showtabs=false,                  
	tabsize=2
}
\lstset{style=mystyle}
\newcommand{\Sref}[1]{Section~\ref{#1}}
\newtheorem{hyp}{Hypothesis}

\title{Day 1, Morning Session: \\ {Sets, Statements, and Proof}}
\author{Name}
\institute{\href{}{PhD Student/Post-Doc}, Trinity College Dublin}
\subtitle{Math and Code Bootcamp}
\date{September 2020}
\begin{document}
\frame{\titlepage}


\begin{frame}{Roadmap for the week}
	

	\begin{itemize}
		
		\item
		We're learning how to make inferences about a population from a sample \vspace{.25cm}
	%	\item
	%	We have characterized a single variable (i.e. $\bar{y}$, $\hat{\pi}$, CIs, etc.) \vspace{.25cm}
		\item \underline{Last time:} We figured out how to determine if two samples are different or independent (diff-in-means, contingency tables)
	\end{itemize}
	
	\vspace{.25cm}
	
	
	\vspace{.25cm}
	\textbf{Outline for today}: \vspace{.25cm}
	\begin{itemize}
		\item From scatterplots to correlations		\vspace{.25cm}
		\begin{itemize}
			\item i.e. how similar are the data (does variation in one explain variation in the other)
		\end{itemize}
		\vspace{.25cm}
		\item Bivariate regression 		\vspace{.25cm}
		\begin{itemize}
			\item Assumptions\vspace{.25cm}
			\item Estimation (i.e. drawing the ``best'' line through data)
		\end{itemize}

	\end{itemize}
	
\end{frame}

\begin{frame}{Step 1: Standardizing variation in variables}


\Large{$$ \frac{x-\bar{x}}{s} $$}


\large	Example: Populations of New England states
\begin{center}\footnotesize{
		\begin{tabular}{ccc}
			\\[-1.8ex]\hline  \\[-1.8ex] 
			\hline \\[-1.8ex]
			& $x$ & $\frac{x-\bar{x}}{s}$ \\ 
			\\[-1.8ex]\hline 
			\\[-1.8ex]
			CT & 3.5mil &\textcolor{red}{?}\\
			ME & 1.3mil &\textcolor{red}{?}\\
			MA & 6.6mil & \textcolor{red}{?}\\
			NH & 1.3mil &\textcolor{red}{?}\\
			RI & 1.0mil &\textcolor{red}{?}\\
			VT & 0.6mil &\textcolor{red}{?} \\
			\\[-1.8ex]\hline  \\[-1.8ex] 
			\hline \\
		\end{tabular}
}\end{center}
\end{frame}

\begin{frame}{Standardizing variables: Mean and SD in \texttt{R}}

\lstinputlisting[language=R, firstline=1,lastline=4]{../code/bc_lecture_d1.R} 
\vspace{.5cm}
\texttt{[1] 2.38 2.3}
\vspace{.5cm}

\Large{$$\bar{x}=2.38 \hspace{10pt} s=2.30$$}

\end{frame}


\begin{frame}{Wrap-up}
	
	\begin{block}{Today we learned about...}	
		\begin{itemize}
			\item Correlations  \vspace{.25cm} 
			\item Simple linear regression: \vspace{.25cm}
			\begin{enumerate}
				\item Assumptions \vspace{.25cm}
				\item Estimation
			\end{enumerate}
			
		\end{itemize}
		
		
	\end{block}
\end{frame}


\end{document}
